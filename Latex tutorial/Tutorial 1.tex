\documentclass[letter]{article}
	% basic article document class
	% use percent signs to make comments to yourself -- they will not show up.

\usepackage{changepage}
\usepackage{enumerate}
\usepackage{amsmath}
\usepackage{amssymb}
\usepackage{amsthm}
\usepackage{mathtools}
	% packages that allow mathematical formatting

\usepackage{graphicx}
	% package that allows you to include graphics
	
\usepackage{pgfplots}


\usepackage{setspace}
	% package that allows you to change spacing

\onehalfspacing
	% text become 1.5 spaced

\usepackage{fullpage}
	% package that specifies normal margins
	
%\makeatletter
%\newcommand{\xRightarrow}[2][]{\ext@arrow 0359\Rightarrowfill@{#1}{#2}}
%\makeatother

\newenvironment{problem}[2][Problem]{\begin{trivlist}
\item[\hskip \labelsep {\bfseries #1}\hskip \labelsep {\bfseries #2.}]}{\end{trivlist}}

\DeclareMathOperator{\SPAN}{span}
\DeclareMathOperator{\HESS}{Hess}
\DeclareMathOperator{\DIAM}{diam}
\newcommand{\PD}[2][]{\frac{ \partial {#1} }{ \partial {#2} } }
\newcommand{\BVEC}[1]{\boldsymbol{#1}}
\newcommand{\CONTRA}{\Rightarrow\!\Leftarrow}


%hyperref-compatible custom float tagging. Place command before caption
\makeatletter \newcommand{\floattag}[1]{
   \@namedef{the\@captype}{#1}%
   \@namedef{theH\@captype}{#1}%
   \addtocounter{\@captype}{-1}} 
\makeatother


\usepackage{graphicx}
\graphicspath{ {C:/Users/jdewh/OneDrive - The University of Chicago/Third Year/PHYS 211/Jupyter Notebooks/} }
\usepackage{lipsum}

\newcommand{\X}{\mathbf{x}}
\newcommand{\Y}{\mathbf{y}}
\newcommand{\BV}[1]{\hat{\mathbf{#1}}}
\newcommand{\IP}[1]{\langle #1 \rangle}

%a ~ can replace a space in text mode to prevent a line break at that space.

\begin{document}
	% line of code telling latex that your document is beginning

%Title section
\begin{center}
	{\large PHYS 211 Latex Tutorial}
	
	Jack Dewhurst \hspace{1cm}
	8 October 2021
	
	\vspace{5mm}
\end{center}
	
%end title section

\section{A Wild Graph Appears!}

\subsection{The Plot Thickens}

\begin{figure}[h]
    \centering
    \includegraphics[scale=.5]{C:/Users/jdewh/OneDrive - The University of Chicago/Third Year/PHYS 211/Q1/Jupyter Notebooks/Example1_Figure1.pdf}
    \floattag{(\textasteriskcentered)}
    \caption{An example figure generated by the Python tutorial. The figure is tagged with an asterisk because I wanted to figure out how to do that.}
    \label{fig:ex1}
\end{figure}

In figure \ref{fig:ex1} above, we can see an example of the kind of data visualization Python allows us to perform. \lipsum[1]


\section{Why Seven Ate Nine}

    A Riemannian metric on a surface $S$ allows us to define a quadratic form $I_p : T_p(S) \to \mathbb{R}$ by
\begin{align} \label{eqn:FirstForm}
I_p (w) = \IP{w,w}_p,
\end{align}
called the \textit{first fundamental form of $S$}. Notice that $\sqrt{I_p}$ defines a norm on $T_p(S)$, so for a vector $w \in T_p(S)$ we write $|w| = \sqrt{I_p(w)}$. 

    Using the first fundamental form,\textsuperscript{(\ref{eqn:FirstForm})} we can define various metric properties on abstract surfaces. 

The \textit{arc length} of a curve $\alpha$ in $S$ is the value
\begin{align}\label{eqn:arcLength}
L = \int_I \sqrt{I_p (\alpha^\prime (t))} \, dt.
\end{align}

    Recalling that in a parametrization $\X$, $\alpha^\prime$ can be expressed $\alpha^\prime (t) = u^\prime(t) \BV{u} + v^\prime(t) \BV{v}$, we can rewrite the arc length\textsuperscript{(\ref{eqn:arcLength})} as
\begin{align}
L = \int_I \sqrt{(u^\prime)^2 E + 2 u^\prime v^\prime F + (v^\prime)^2 G} \, dt
\end{align}
where $E, F$, and $G$ are evaluated at $\alpha(t)$. We also say that a curve $\alpha: I \to S$ is \textit{parametrized by arc length} if $|\alpha^\prime (t)| = 1$ for all $t \in I$. \lipsum[2]


\section{Your Table is Right This Way}

\begin{table}[h] 
\centering

\begin{tabular}{ | c c | }
\hline
Symbol & Value \\ [.25em]
\hline \hline
$\gamma$ & $\frac{1 + \sqrt{5}}{2}$ \\ [.25em]
$\gamma^{-1} $ & $\frac{\sqrt{5} - 1}{2}$ \\[.25em]
\hline 
\end{tabular}
\caption{Values of the golden ratio}
\label{tab:gamma}
\end{table}

Table \ref{tab:gamma} is an example of a table in \LaTeX. \lipsum[3]



\end{document}
	% line of code telling latex that your document is ending. If you leave this out, you'll get an error
